\newcommand{\toolname}{{\sc Faery}\xspace}
\newcommand{\neo}{{\sc Neo} \xspace}

\newcommand{\semantics}[1]{\llbracket{#1}\rrbracket}
\newcommand{\asemantics}[1]{\llbracket{#1}\rrbracket^{\#}}
\newcommand{\strategy}{\mathcal{S}}

\newcommand{\oracle}[1]{\llbracket{#1}\rrbracket}
\newcommand{\lang}{\mathcal{L}}
\newcommand{\user}{\mathcal{O}}
\newcommand{\mut}{\mu}
%\newcommand{\example}{\mathcal{Q}} this is defined to be something else 
\newcommand{\abs}{\Psi}
\newcommand{\hole}{\square}

\usepackage{xcolor}
\usepackage{bbm}

\usepackage{float}
\newfloat{algorithm}{t}{lop}

\newtheorem{theorem}{Theorem}[section]
\newtheorem{conjecture}[theorem]{Conjecture}
\newtheorem{proposition}[theorem]{Proposition}
\newtheorem{lemma}[theorem]{Lemma}
\newtheorem{corollary}[theorem]{Corollary}
%\newtheorem{example}[theorem]{Example}
\newtheorem{definition}[theorem]{Definition}
\newtheorem{assumption}[theorem]{Assumption}
\newtheorem{sample}[theorem]{Example}



\newcommand{\lattice}{\mathcal{P}}
\newcommand{\element}{P}
\newcommand{\refines}{\sqsubseteq}
\newcommand{\refinedby}{\sqsupseteq}
\newcommand{\aval}{\phi}
\newcommand{\prunes}{\not\models}
\newcommand{\maxprunes}{\not\models^0}
\newcommand{\example}{e}
\newcommand{\examples}{\vec{e}}
